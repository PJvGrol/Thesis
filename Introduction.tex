\chapter{Introduction}
This thesis has been inspired by a recent paper that introduced OmniLedger \cite{OmniLedger}: a scalable, decentralized ledger. In a world where more and more uses are found for blockchain technology, the promise of such a technology that scales in performance with the amount of users is incredibly interesting. Considering that current systems might actually perform worse when the amount of users increases, it is interesting to look at OmniLedger to see whether it can help in that regard.

In particular, this thesis focuses on multi-signature algorithms that may be used to improve the performance of RandHound \cite{RandHound}. RandHound is used by OmniLedger to ensure that the system remains uncompromised. Evaluation has shown that RandHound takes up more than 70\% of the total runtime in the bootstrap process of OmniLedger \cite{OmniLedger}. Since this process happens periodically, improving the performance of RandHound should lead to performance improvement of OmniLedger. 

Since RandHound uses a multi-signature scheme as part of its algorithm, which constitutes a large part of the running time of the algorithm as a whole, this thesis focuses on improving RandHound by improving the multi-signature algorithm used.

Furthermore, OmniLedger also makes use of CoSi \cite{CoSi} in the consensus protocol, which also makes use of multi-signatures, so the same performance improvement can be achieved there by improving the multi-signature algorithm.
\\
\\
Therefore the research question of this thesis is as follows: "Which multi-signature scheme is most suitable for performance improvement of OmniLedger". To answer this question, this thesis researches the performance and security requirements of two multi-signature schemes. MuSig \cite{SchnorrMulti} is based on Schnorr signature schemes and BLS multi-signature \cite{BLSMulti} is based on the BLS \cite{BLS} signature scheme.

To properly answer these questions some mathematical and technical background is needed.

The second chapter, Mathematical Background, introduces algebraic varieties, leading up to an explanation of elliptic curves and the Weil pairing on elliptic curves. It further introduces the discrete logarithm problem and the elliptic curve discrete logarithm problem. Elliptic curves, the Weil pairing and the (elliptic curve) discrete logarithm problem are used to great effect in cryptography.

The third chapter, Technical Background, then introduces the technological background of ledgers, blockchain and multi-signatures among other terms and concepts used to clearly show the context of the research.

Chapter four, Related Work, discusses related works, such as the work on OmniLedger \cite{OmniLedger}, RandHound \cite{RandHound}, MuSig \cite{SchnorrMulti} and BLS \cite{BLSMulti} and its multi-signature variant \cite{BLSMulti}. It further explains the contribution of this thesis to the existing works.

In chapters five, BLS Multi-Signature Scheme, and six, Schnorr Multi-Signature Scheme, the two multi-signature schemes are studied.

This study leads chapter seven, Comparison and Conclusion, in which the two schemes are compared and the research question is answered.