\chapter{Boneh-Lynn-Shacham Multi-Signature Scheme} \label{BLSMulti}
\section{Boneh-Lynn-Shacham Signature Scheme}
The Boneh-Lynn-Shacham \cite{BLS} (BLS) signature scheme, is a signature scheme utilising the Weil pairing to create signatures. The scheme uses the following:
\begin{itemize}
	\item A bilinear pairing $e:\mathbb{G}_0 \times \mathbb{G}_1 \to \mathbb{G}_T$ which is efficiently computable and non-degenerate. The three groups have prime order $q$ and the generators of $\mathbb{G}_0$ and $\mathbb{G}_1$ are $g_0$ and $g_1$, respectively.
	\item A hash function $H_0:M\to \mathbb{G}_0$ to hash the message $M$.
\end{itemize}
The scheme then has three phases. In the first phase the key is generated, in the second, the message is signed and in the final phase, the signature is verified.
\begin{itemize}
	\item To generate the key, $\alpha \in \mathbb{Z}_q$ is picked randomly and then used to create $h=g_1^\alpha\in\mathbb{G}_1$. Here $\alpha$ is the private key, and $h$ the public key.
	\item The message $m$ is then signed by creating $\sigma=H_0(m)^\alpha\in\mathbb{G}_0$. The result is therefore a single group element.
	\item The signature is verified if $e(g_1,\sigma)=e(h,H_0(m))$ holds, otherwise the signature is a forgery.
\end{itemize}
To see that the verification holds, note that $e$ is bilinear, and as such $e(g^x,g^y)=e(g,g)^{xy}$. Therefore it follows that $e(g_1,\sigma)=(g_1,H_0(m)^\alpha)=(g_1,H_0(m))^\alpha=e(h,H_0(m))$ and the signature is verified.
\section{Algorithm}
Boneh, Drijvers and Neven \cite{BLSMulti} noted that the intuitive multi-signature algorithm using BLS, which aggregates signatures $\sigma_1,\dots,\sigma_n$ by computing $\sigma=\sigma_1\dots\sigma_n$ can be easily attacked. An attacker can register his public key $h_2$ by using $h_1$ of another user, called Alice such that $h_2=g_1^\beta\cdot h_1^{-1}$, where the attacker picks $\beta\in\mathbb{Z}_q$ himself.

The attacker can now present the signature $\sigma=H_0(m)^\beta$ for some message $m$ and claim that it was signed by both him and Alice since it follows that $e(g_1,\sigma)=e(g_1,H_0(m)^\beta)=e(g_1^\beta,H_0(m))=e(h_1\cdot h_2,H_0(m))$, and that therefore the signature is genuine.

The BLS multi-signature scheme that is resistant against this type of attack works slightly different, but uses the same principles. As added prerequisite, the BLS multi-signature scheme requires a second hash function $H_1:\mathbb{G}_1^n\to R^n$ with $R^n=\{2^0,\dots,x^128\}$.

Key generation and message signing remain the same as in single use BLS, but the aggregation scheme works differently.
\begin{itemize}
	\item The function $H_1$ is used to calculate $(t_1,\dots,t_n)=H_1(h_1,\dots,h_n)\in R^n$. These values are used to create the aggregate signature $\sigma=\sigma_1^{t_1}\dots\sigma_n^{t_n}\in\mathbb{G}_0$.
	\item The signature $\sigma$ is verified by computing $(t_1,\dots,t_n)=H_1(h_1,\dots,h_n)\in R^n$ and $h=h_1^{t_1}\dots h_n^{t_n}\in\mathbb{G}_1$. The signature is genuine if $e(g_1,\sigma)=e(h,H_0(m))$.
\end{itemize}
Because the multi-signature now requires genuine input from each participant, the attack on the previous scheme is no longer possible.
\section{Security}
The security of this scheme is based on the computational Diffie-Hellman assumption (co-CDH) \cite{CDH}. This assumption, closely related to the discrete logarithm problem, is as follows.
\begin{defn}
	Take group $\mathbb{G}$ of order $q$. Given $(g,g^a,g^b)$ with $g$ a random generator and $a,b\in\{0,\dots,q-1\}$, it is infeasible to compute $g^{ab}$
\end{defn}
The security proof shows that any attacker that manages to forge a message in the BLS multi-signature scheme, can use that very same algorithm to break co-CDH, which contradicts the statement that it is infeasible to compute.
\section{Performance}
Since aggregation of the signatures only requires the individual signatures and the value $H_1(h_i)$ for signer $i$, the aggregation can happen as soon as all needed signatures are published. Specifically, this means that it is not needed for all signers to participate in the process at the exact same time, completely skipping any multi-round protocol between signers.

Furthermore, the BLS multi-signature algorithm is set up in such a way that the aggregated public key can be computed without knowing the message $m$, since it only uses public values to calculate the key. This, of course, means that, if it is known who will be the signers, any party wishing to verify the signature can calculate the aggregate public key beforehand, allowing for greater performance.

Adding to this, the scheme also allows to check a set of aggregate signatures, to allow batch verification over different messages. Considering distinct messages $m_1,\dots,m_b$ with corresponding signatures $\sigma_1,\dots,\sigma_b$, calculating $\sigma=\sigma_1\cdot\sigma_b$ allows to verify that $e(g_1,\sigma)=e(h_1,H_0(m_1))\cdots e(h_b,H_0(m_b))$.