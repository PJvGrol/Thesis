\documentclass[12pt]{report}
\usepackage{amssymb}
\usepackage{amsthm}
\usepackage[backend=bibtex8]{biblatex}
\usepackage[utf8]{inputenc}
\usepackage{mathtools}

\theoremstyle{plain}
\newtheorem{thm}{Theorem}[chapter] % reset theorem numbering for each chapter
\theoremstyle{definition}
\newtheorem{defn}[thm]{Definition} % definition numbers are dependent on theorem numbers
\newtheorem{exmp}[thm]{Example} % same for example numbers

\addbibresource{references.bib}
\begin{document}
	\chapter{Technical Background}
	This chapter provides the technical background to the various terms and concepts used in this thesis. If the reader is already familiar with anything explained in a section in relation to blockchain, skipping it should not hinder in further reading. If the reader is unfamiliar, or wishes to refresh his memory, reading this chapter is advised.
	
	A bottom-up approach is used in this chapter, which means that sections may use or require terms and concepts explained in a previous section, or previous sections. Each section aims to give an intuitive understanding via an example and offers a detailed explanation. A visual representation of how the various sections are linked can be found in appendix A.
	
	The chapter first leads up to explaining blockchain, followed by an introduction to signature schemes. To end the chapter, RandHound and finally OmniLedger are explained.
	
	\section{Concensus protocol}
	By definition consensus is an opinion that everyone in a group agrees with or accepts. An agreement, in other words. A protocol, in the technical definition, is an established method for connecting computers so that they can exchange information. A consensus protocol is therefore a method to reach agreement between connected computers.
	
	In relation to blockchain technology, a consensus protocol\cite{ibmconsensus} is a protocol used to reach agreement over input to add to the ledger, and the order in which it should be added. Blockchain technology recognises various such protocols, operating in various ways under different assumptions and requirements.
	
	Below brief explanations will be given of the consensus protocols used by Bitcoin and Stellar. The former uses a concept known as proof-of-work, the latter introduces a notion of trust.
	
	\subsection{Bitcoin}
	Conversely, the consensus protocol used by Bitcoin\cite{bitcoinconsensus} does not require any trust in other participants. Instead it relies on a mechanism known as proof-of-work. In Bitcoin a reward goes to the participant that manages to create the next agreement. This agreement is a set of transactions, called a block, that fulfills the requirements set for the block. Once a participant has found such a block, he sends it to as many others as he can, so that his result will be on the chain, which makes the reward his. Other participant can easily verify that a block meets those requirements, after which they'll accept the result and forward it to others.
	
	The protocol functions with the help of a one-way hash function. A one-way function is a function that makes it easy to compute f(x) given x, but makes calculating x from f(x) (near) impossible. A one-way hash function has the added property of producing output of a fixed length.
	
	What happens in the protocol used by Bitcoin is that each participant will create a block. This block then serves as input for the one-way hash function. The goal is to have a block that results in a hash with a certain amount of leading zero's. Given the block, others can easily verify that hashing the block results in a hash with a sufficient amount of leading zero's.
	
	Producing a wrong block, therefore, is not in the best interest of the participant. Others will quickly refure the block, and all the effort put into creating the block will be wasted.
	
	Therefore the notion of proof-of-work for the Bitcoin consensus protocol is apt. Arriving at a sufficient result requires the participant to try various input blocks, with no way to compute one from the desired result.
	
	\subsection{Stellar}
	To understand the Stellar Consensus Protocol\cite{stellarconsensus} (SCP) some definitions are needed. SCP introduces a quorum, a set of participants that is sufficient to reach agreement, and a quorum slice, a subset of a quorum that can convince one participant of the agreement.
	
	The basis of SCP is that participant trust others in their quorum slice to behave honestly, hence the notion of trust in SCP. Assuming all participant behave honestly and under the same set of input and rules, each participant should arrive at the same conclusion. After a participant reaches his conclusion, he publishes it to the others in his quorum slice. In publishing his conclusion he votes for it, as do all others in his slice. The final conclusion is the one most voted for.
	
	As long as there are sufficient reputable participants, the network can never be corrupted. Behaving dishonestly is further disincentivised by the protocol, since those participants are not trusted to be part of a quorum slice, and the influence of their opinion will then quickly be reduced. Conversely, reputable participants will be part of multiple quorum slices, which gives them a large influence over the decisions made.
	
	To ensure that the whole network is indeed connected and reaches network wide consensus, SCP requires that there are no disjunct quorums. Since disjunct quorum can reach their opinion separate from the network, a different agreement undermines the network wide consensus. Participants wishing to join the network have to make sure that they join a quorum in such a way that no disjunct quorums are created.
	
	\section{Distributed Ledger}
	By definition, a distributed ledger\cite{distributedledger} (DL) would be shared records. This remains true in the context of blockchain technology, but that is not all that it is. Although in use it acts as if there is one ledger that everybody uses, so that everybody agrees what has happened, the technique behind it works differently.
	
	The most important property of a DL in the context of blockchain is the irrefutable state of truth it represents. Anything on the distributed ledger has happened exactly as described there, and no-one can alter those existing records. A DL may therefore be of interest to any group of parties looking to abolish ledger conflicts, to be able to trust that anything in the ledger has indeed happened, nothing more, nothing less, nothing else.
	
	In a DL system, each participant holds and keeps track of his own copy of the ledger. This happens independently, so ledger states are never directly compared. The mechanism used by the sytem ensures that each participants holds the exact same conclusion on his ledger, thereby assuring the irrefutable state of truth.
	
	What happens in the system, is that each participant starts at the same basis: an empty ledger. Input is then presented to all participants, who will handle the input as the system describes. Having handled the input and produced the result, a consensus protocol is used to ensure network wide agreement. Once agreement has been reached, each participant adds input to his ledger as agreed upon, ensuring that each participant reaches the same next ledger state.
	
	Since it may happen that participants join at a later time, mechanisms exist to ensure that the participant can catch-up to the current ledger. This is not done by presenting the ledger, but instead by presenting the input for each ledger state, and corresponding consensus. The new participant then uses that information to construct his ledger from the basis state up, eventually reaching the same ledger state as all other participants.
	
	Blockchain is a form of a DL, using blocks of transactions to shape the ledger and various consensus protocols to agree upon blocks to be added. Since this thesis focusses on techniques used in blockchain, any references to a chain, ledger or anything similar will mean a blockchain, a specific form of a DL.
\end{document}
