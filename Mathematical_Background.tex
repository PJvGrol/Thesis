\chapter{Mathematical Background}
This chapter provides the mathematical background to elliptic curve cryptography. If the reader is already familiar with elliptic curves and the Weil pairing, skipping to the section on elliptic curve cryptography should not hinder the understanding of that chapter. If the reader is unfamiliar, or wishes to refresh his memory, reading this chapter is advised.

A background in and understanding of fields, rings and Galois groups is assumed in this chapter.

This chapter will start with algebraic varieties and lead up to elliptic curves and the Weil pairing, finishing with an explanations of the (elliptic curve) discrete logarithm problem and the Schnorr group.

As source material in this chapter, the excellent work of Joseph H. Silverman is used \cite[Chapter 1-3]{EllipticCurvesBook}, along with some of his slides \cite{EllipticCurvesSlides}.

\section{Algebraic Varieties}
\subsection{Notation}
Before explaining algebraic varieties, the following notation is set.
\begin{itemize}
	\item $K$ is a perfect field.
	\item $\bar{K}$ is a fixed algebraic closure of $K$.
	\item $\bar{K}[X]=\bar{K}[X_1,\dots,X_n]$ is a polynomial ring in $n$ variables.
	\item $\bar{K}/K$ is a field extension. This means that $\bar{K}$ contains $K$ and the operations of $K$ are the operations of $\bar{K}$ restricted to $K$.
	\item $G_{\bar{K}/K}$ is the Galois group of $\bar{K}/K$.
\end{itemize}

\subsection{Affine Varieties}
The following definition defines Cartesian $n$-space and the subsets that are defined by zeros of polynomials.
\begin{defn}
	Cartesian, also known as affine, $n$-space, which is implied to be over $K$, is the set of $n$-tuples.
	\begin{equation*}
	\mathbb{A}^n=\mathbb{A}^n(\bar{K})=\{P=(x_1,\dots,x_n):x_i\in\bar{K}\}
	\end{equation*}
	This leads to the set of $K$-rational points of $\mathbb{A}^n$ as the following set.
	
	\begin{equation*}\mathbb{A}^n(K)=\{P=(x_1,\dots,x_n)\in\mathbb{A}^n:x_i\in K\}
	\end{equation*}
\end{defn}

From this point on, any space used is, indeed, an affine space. Using the Galois group $G_{\bar{K}/K}$ allows for an alternative description of $\mathbb{A}^n(K)$, because the Galois group fixes elements $P\in\mathbb{A}^n$.
\begin{rem}
	Considering the Galois group $G_{\bar{K}/K}$ leads to an action on $\mathbb{A}^n$ such that for $\sigma\in G_{\bar{K}/K}$ and $P\in\mathbb{A}^n$, it leads to
	\begin{equation*}
	P^\sigma=(x_1^\sigma,\dots,x_n^\sigma)
	\end{equation*}
	This then allows to define $\mathbb{A}^n(K)$ as
	\begin{equation*}
	\mathbb{A}^n(K)=\{P\in\mathbb{A}^n:P^\sigma=P,\forall\sigma\in G_{\bar{K}/K}\}
	\end{equation*}
\end{rem}

Using ideals in $\bar{K}[X]$ leads to the construction of an algebraic set on $\mathbb{A}^n$.
\begin{defn}
	Taking $I\subset\bar{K}[X]$ as an ideal, each of these can be associated with a subset of $\mathbb{A}^n$ such that
	\begin{equation*}
	V_I=\{P\in\mathbb{A}^n:f(P)=0,\forall f\in I\}
	\end{equation*}
	Any set of this form is an algebraic set.
\end{defn}

Hilbert's Nullstellensatz \cite{Hilbert} allows to link algebraic sets to ideals, which defines a relationship between geometry and algebra.

\begin{defn}[Nullstellensatz]
	Given the algebraic set $V_I$ and a polynomial $p\in\mathbb{A}^n$. If
	\begin{equation*}
	p(x)=0\text{, }\forall x\in V_I \implies \exists r\in\mathbb{N}:p^r\in I
	\end{equation*}
	That is to say, if the polynomial vanishes on the algebraic set, there is some natural number $r$ such that $p^r$ lies in the ideal $I$.
\end{defn}

On an algebraic set, in turn, the ideal may be defined.
\begin{defn}
	Given an algebraic set V, the ideal of V is
	\begin{equation*}
	I(V)=\{f\in\bar{K}[X]:f(P)=0,\forall P\in V\}
	\end{equation*}	
\end{defn}
If $I(V)$ can be generated by polynomials in $K[X]$ for an algebraic set $V$, it is defined over $K$ and noted as $V/K$.

Using an algebraic set $V$ and the corresponding ideal $I(V)$ leads to the definition of an affine variety.
\begin{defn}
	$V$ is called an affine variety if $I(V)$ is a prime ideal in $\bar{K}[X]$.
\end{defn}

\begin{exmp}
	The set
	\begin{equation*}
	\{(x,y,x):x^2+y^2-z^2=0\}
	\end{equation*} is an affine variety, representing the cone, which can be denoted by $V(x^2+y^2-z^2)$. This set can be expanded by further requiring that $ax+by+cz=0$, leading to another affine variety
	\begin{equation*}
	\{(x,y,z):x^2+y^2-z^2=0,ax+by+cz=0\}
	\end{equation*}
\end{exmp}

\subsection{Projective Varieties}
Making an affine space a projective space is done by adding the so-called ``points at infinity". This same process can be recreated to define projective varieties.
\begin{defn}
	$\mathbb{P}^n=\mathbb{P}^n(\bar{K})$ denotes the projective $n$-space over $K$ and is constructed as the set of all $(n+1)$ tuples
	\begin{equation*}
	(x_0,\dots,x_n)\in\mathbb{A}^{n+1}
	\end{equation*}
	
	These tuples must be such that at least one element is non-zero modulo the following equivalence relation
	\begin{equation*}
	(x_0,\dots,x_n)\sim(y_0,\dots,y_n)
	\end{equation*}
	
	if there is a $\lambda\in\bar{K}^*$ such that $x_i=\lambda y_i,\forall i$. The resulting equivalence class
	\begin{equation*}
	\{(\lambda x_0,\dots,\lambda x_n):\lambda\in\bar{K}^*\}
	\end{equation*}
	
	is noted as $[x_0,\dots,x_n]$, with the elements being called homogeneous coordinates for the point $P\in\mathbb{P}^n$ that corresponds to it. This leads to the set of $K$-rational points in $\mathbb{P}^n$
	\begin{equation*}
	\mathbb{P}^n(K)=\{[x_0,\dots,x_n]\in\mathbb{P}^n:x_i\in K\}
	\end{equation*}
\end{defn}
As with affine varieties, the Galois group $G_{\bar{K}/K}$ allows for an alternate definition of $\mathbb{P}^n(K)$.
\begin{rem}
	The Galois group $G_{\bar{K}/K}$ acts on $P\in\mathbb{P}^n$ via
	\begin{equation*}
	[x_0,\dots,x_n]^\sigma=[x_0^\sigma,\dots,x_n^\sigma]
	\end{equation*}
	and as such it leads to
	\begin{equation*}
	\mathbb{P}^n(K)=\{P\in\mathbb{P}^n:P^\sigma=P,\forall\sigma\in G_{\bar{K}/K}\}
	\end{equation*}
\end{rem}

With the existence homogeneous coordinates, homogeneous polynomials and ideals can be defined.
\begin{defn}
	A polynomial $f\in\bar{K}[X]$ is homogeneous of degree $d$ if
	\begin{equation*}
	f(\lambda X_0,\dots,\lambda X_n)=\lambda^df(X_0,\dots,X_n),\forall \lambda\in\bar{K}
	\end{equation*}
	An ideal $I\subset\bar{K}[X]$ is homogeneous if it is generated by homogeneous polynomials.
\end{defn}

With $f$ a homogeneous polynomial, it is interesting to look at points $P\in\mathbb{P}^n$ where $f(P)=0$. Using these points, for each homogeneous ideal $I$, a subset can be defined.
\begin{defn}
	Taking $f$ to be a homogeneous polynomial, any set of the following form for a homogeneous ideal $I$ is a projective algebraic set.
	\begin{equation*}
	V_I=\{P\in\mathbb{P}^n:f(P)=0,\forall f\in I\}
	\end{equation*}
\end{defn}

Such a set has, of itself, a homogeneous ideal.
\begin{defn}
	$I(V)$, the homogeneous ideal of $V$ is the ideal of $\bar{K}[X]$ that is generated with homogeneous $f$ by
	\begin{equation*}
	\{f\in\bar{K}[X]:f(P)=0,\forall P\in V\}
	\end{equation*}
\end{defn}

The homogeneous ideal is used to define a projective variety.
\begin{defn}
	If the homogeneous ideal $I(V)$ of projective algebraic set $V_I$ is a prime ideal in $\bar{K}[X]$, the set $V_I$ is called a projective variety.
\end{defn}

\section{Algebraic Curves}
Before explaining algebraic curves, the following notation is set, with $C$ a curve and $P\in C$ a point on it. The work in \cite{RRBrouwer} was used as additional source for an explanation of the Riemann-Roch theorem.
\subsection{Notation}
\begin{itemize}
	\item $C/K$ means that $C$ is defined over $K$.
	\item $\bar{K}(C)$ is the function field of $C$ over $\bar{K}$.
	\item $K(C)$ is the function field of $C$ over $K$.
	\item $\bar{K}[C]_P$ is the local ring of $C$ at $P$.
	\item $M_P$ is the maximal ideal of $\bar{K}[C]_P$.
\end{itemize}
\subsection{Curves}
Algebraic curves are projective varieties of dimension one. Elliptic curves, in turn, are algebraic curves of genus one. As such it is important to define the genus of an algebraic curve, which is done by the way of the Riemann-Roch theorem.

Every curve $C$ has a corresponding divisor group $Div(C)$, which is the free abelian group generated by the points of C. A divisor $D\in Div(C)$ is defined as a sum.
\begin{defn} Given $n_P\in\mathbb{Z}$ and $n_P\neq 0$ for finitely many $P\in C$
	\begin{equation*}
	D=\sum_{P\in C}n_P(P)
	\end{equation*}
\end{defn}

The degree of the divisor is determined by the sum of non-zero $p_n$.
\begin{defn}
	\begin{equation*}
	deg(D)=\sum_{P\in C}n_p
	\end{equation*}
\end{defn}

It is quite easy to see that the divisors form an abelian group under regular addition, since they are defined as sums. The degree of the divisor actually defines a homomorphism to $\mathbb{Z}$.
\\
\\

For a function $f\in\bar{K}(C)$ the principal divisor $(f)$ is defined as
\begin{defn}
	\begin{equation*}
	(f)=\sum v_P(f)P
	\end{equation*}
\end{defn}
Here $P\in C$, and $v_P=\#\text{zeros}-\#\text{poles}$ of $f$ at $P$.
\newpage
The principal divisors form a subgroup of the divisors and lead to the definition of the Picard group:
\begin{defn}[Picard group]
	\begin{equation*}
	Pic(C)=\{Div(C)\}/\{\text{principal divisors}\}
	\end{equation*}
\end{defn}
The Picard group is therefore the quotient of $Div(C)$ by its subgroup of principal divisors.
\\
\\
Given a divisor $D$, a set of functions $\mathcal{L}(D)$ can be associated to it such that
\begin{equation*}
\mathcal{L}(D)=\{f\in\bar{K}(C)^*:div(f)\geq-D\}\cup\{0\}
\end{equation*}

This set $\mathcal{L}(D)$ is a $\bar{K}$-vector space of finite dimensions. This dimension is noted by
\begin{equation*}
\ell(D)=\text{dim}_{\bar{K}}\mathcal{L}(D)
\end{equation*}
\\
\\
To define the canonical divisor, which is the last bit of knowledge needed to understand the Riemann-Roch theorem, the space of differential forms on $C$, noted by $\Omega_C$ must first be defined.

\begin{defn}[Space of differential forms]
	$\Omega_C$ is the $\bar{K}(C)$-vector space generated by $dx$ for $x\in\bar{K}(C)$ such that:
	\begin{enumerate}
		\item $d(x+y)=dx+dy$ for all $x,y\in\bar{K}(C)$
		\item $d(xy)=x\text{ }dy+y\text{ }dx$ for all $x,y\in\bar{K}(C)$
		\item $d(a)=0$ for all $a\in\bar{K}$
	\end{enumerate}
\end{defn}
In other words, functions must be differentiable according to the common rules on $\mathbb{R}$.
\\
\\
The canonical divisor is, for $\omega\in\Omega_C$
\begin{defn}[Canonical divisor]
	\begin{equation*}
	(\omega)=\sum v_P(\omega)P
	\end{equation*}
\end{defn}
Or simply a divisor for a function in differential form.
Now the genus of the curve can be defined by the way of the Riemann-Roch theorem.
\begin{thm}[Riemann-Roch] \label{RiemannRoch}
	Let $C$ be a smooth curve and let $K_C$ be a canonical divisor on $C$. There is an integer $g\geq0$, called the genus of $C$, such that for every divisor $D\in Div(C)$,
	\begin{equation*}
	\ell(D)-\ell(K_C-D)=deg(D)-g+1
	\end{equation*}
\end{thm}
To rewrite this specifically for the genus of the curve, the following equation is found.
\begin{equation*}
g=deg(D)+\ell(K_C-D)-\ell(D)+1
\end{equation*}
Proof for the Riemann-Roch theorem is found in \cite{RiemannRoch}.


\section{Elliptic Curves}
This section introduces elliptic curves and the group action on elliptic curves. Furthermore it defines and explains the Weil pairing on elliptic curves. The section is concluded with an explanation of elliptic curve cryptography.
\subsection{Weierstrass Equations}
The Weierstrass equation for elliptic curves is written as as follows
\begin{defn}
	Given a base point $O$ and $a_1,\dots,a_6\in\bar{K}$ the Weierstrass equation for an elliptic curve is, for variables $X,Y,Z$:
	\begin{equation}
	Y^2Z+a_1XYZ+a_3YZ^2=X^3+a_2X^2Z+a_4XZ^2+a_6Z^3.
	\end{equation}
	Equation (2.1) is generally written with non-homogeneous coordinates $x=X/Z$ and $y=Y$
	\begin{equation}
	E:y^2+a_1xy+a_3y=x^3+a_2x^2+a_4x+a_6.
	\end{equation}
	If $a_1,\dots,a_6\in K$, $E$ is said to be defined over $K$.
\end{defn}

If $char(\bar{K})\neq 2$, then equation (2.2) can be simplified by substituting
\begin{equation}
y=\frac{1}{2}(y-a_1x-a_3).
\end{equation}

This leads to the following equation
\begin{equation}
E:y^2=4x^3+b_2x^2+2b_4x+b_6
\end{equation}

Here the variables are defined as
\begin{equation}
{b_2=a_1^2+4a_2}
\end{equation}
\begin{equation}
{b_4=2a_4+a_1a_3}
\end{equation}
\begin{equation}
{b_6=a_3^2+4a_6}
\end{equation}

Furthermore define
\begin{equation}
b_8=a_1^2a_6+4a_2a_6-a_1a_3a_4+a_2a_3^2-a_4^2
\end{equation}
\begin{equation}
c_4=b_2^2-24b_4
\end{equation}
\begin{equation}
c_6=-b_2^3+36b_2b_4-216b_6
\end{equation}
\begin{equation}
\Delta=-b_2^2-8b_4^3-27b_6^2+9b_2b_4b_6
\end{equation}
\begin{equation}
j=c_4^3/\Delta
\end{equation}
\begin{equation}
\omega=\frac{dx}{2y+a_1x+a_3}=\frac{dy}{3x^2+2a_2x+a_4-a_1y}
\end{equation}
\\
\\
Using the variables defined in (2.5) through (2.13) is is easy to calculate
\begin{equation*}
4b_8=b_2b_6-b_4^2
\end{equation*}
\begin{equation*}
1728\Delta=c_4^3-c_6^2
\end{equation*}
\\
\\
If furthermore $char(\bar{K})\neq 2,3$, the following substitution can be applied to (2.3)
\begin{equation*}
(x,y)=(\frac{x-3b_2}{36},\frac{y}{108})
\end{equation*}

which reduces (2.3) to the following simple form
\begin{equation}
E:y^2=x^3-27c_4x-54c_6.
\end{equation}
\\
\\
Some of the variables defined in (2.5) through (2.13) have specific meanings to the Weierstrass equations.
\begin{defn}
	The value $\Delta$ is the discriminant of the Weierstrass equation, $j$ is the $j$-invariant of the elliptic curve and $\omega$ is the invariant differential that is associated with the Weierstrass equation.
\end{defn}

\begin{defn}[Invariant differential]
	The invariant differential is named such, for the differential $\omega$ remains invariant under translation. With translation-by-Q map defined as:
	\begin{equation*}
	\tau_Q:E\to E\text{ ,	}P\to P+Q
	\end{equation*}
	it is relatively easy to show, using (2.3.3.1.3), that
	\begin{equation*}
	\tau_Q^*\omega=\omega
	\end{equation*}
	To calculate the proof, write $x(P+Q)$ and $y(P+Q)$ in terms of $X(P,X(Q),Y(P)$ and $Y(Q)$.
\end{defn}

An alternate proof for the invariant differential is given in \cite[page 76]{EllipticCurvesBook}.

The Weierstrass equation in (2.14) can be simplified even further if the characteristic of $K$ is not $2$ or $3$, such that
\begin{equation*}
E:y^2=x^3+Ax+B
\end{equation*}

In this case it follows that
\begin{equation*}
\Delta=-16(4A^3+27B^2)
\end{equation*}
\begin{equation*}
j=-1728\frac{(4A)^3}{\Delta}
\end{equation*}

\subsection{Group Action}
In this section the elliptic curve $E$ is given by a Weierstrass equation. As such, $E\subset\mathbb{P}^2$ consists of points $P=(x,y)$ satisfying
\begin{equation*}
f(x,y)=y^2+a_1xy+a_3y-x^3-a_2x^2-a_4x-a_6=0
\end{equation*}

It is easy to see that the equation has degree three, and that, as such, there are three points of intersection for line $L\subset\mathbb{P}^2$. These points $P,Q,R$ need not be distinct, since $L$ may be a tangent. 
With the knowledge that there are three points of intersection, the composition law $\oplus$ on $E$ is defined as follows.
\begin{defn}
	Let $P,Q\in E$ and let $L$ be the line through $P$ and $Q$ if the points are distinct, or the tangent to $E$ at $P$ if $P=Q$. Let $R$ be the other point of intersection of $L$ with $E$. Denote by $L'$ the line through $R$ and $O$, which intersects $E$ at a third point, denoted by $P\oplus Q$.
\end{defn}
Now the usage of the symbol $\oplus$ is justified by showing that it makes $E$ into an abelian group with identity $O$.
\begin{thm}
	The composition law $\oplus$ has the following properties:
	\begin{enumerate}
		\item If line $L$ intersects $E$ at points $P,Q,R$, not necessarily distinct, then
		\begin{equation*}
		(P\oplus Q)\oplus R=O
		\end{equation*}
		\item $P\oplus O=P\text{\space}\forall P\in E$
		\item $P\oplus Q=Q\oplus P\text{\space}\forall P,Q\in E$
		\item Given $P\in E$ there is a point of $E$, denoted $\ominus P$, such that
		\begin{equation*}
		P\oplus(\ominus P)=O
		\end{equation*}
		\item Given $P,Q,R\in E$
		\begin{equation*}
		(P\oplus Q)\oplus R=P\oplus(Q\oplus R)
		\end{equation*}
		\item Suppose that $E$ is defined over $K$, then the following is a subgroup of $E$
		\begin{equation*}
		E(K)=\{(x,y)\in K^2:y^2+a_1xy+a_3x^2+a_4x+a_6\}\cup\{O\}.
		\end{equation*}
	\end{enumerate}
\end{thm}
Geometrical proofs for the group action are found in appendix A.

\begin{prf}
	\text\linebreak
	\begin{enumerate}
		\item From the definition it follows that the line $L'$ used to construct $P\oplus Q$ intersects $E$ at $R,O$ and $P\oplus Q$. Therefore the intersection point is $O$, and the line through $O$ and $O$ ends up in $O$. In other words, the tangent line to $E$ at $O$ intersects $E$ at $O$ with multiplicity 3.
		\item Note that in this case, the lines $L$ and $L'$ are the same, and that as such the intersection of the line through $O$ and $P\oplus O$ intersects $E$ at precisely $P$.
		\item It is easy to see that the construction of $P\oplus Q$ is symmetric in both $P$ and $Q$. The line through two points that remain fixed does not depend on the order in which the points are picked.
		\item Take a line through $P$ and $O$ and call the intersection point $R$. Using $1.$ and $2.$ it is easy to see that $O=(P\oplus O)\oplus R=P\oplus R.$
		\item To prove associativity, the explicit formulas given in (2.3.3) can be used. A geometrical proof is found in appendix A.
		\item If both $P$ and $Q$ have coordinates in $K$, then the coefficients of the equation of the line $L$ connecting $P$ and $Q$ are in $K$. If, furthermore, $E$ is defined over $K$ then the third point will be a rational combination of coefficients of the line and of $E$, and will thus be in $K$.
	\end{enumerate}
\end{prf}
\begin{rem}
	Since it has been proven that $\oplus$ is a group operation, this thesis will from now on use $+$ instead, and $-$ for $\ominus$. Furthermore, define
	\begin{equation*}
	mP=\overbrace{P+\dots+P}^m \text{if } m>0
	\end{equation*}
	\begin{equation*}
	mP=\overbrace{-P-\dots-P}^m \text{if } m<0
	\end{equation*}
	\begin{equation*}
	0P=O
	\end{equation*}
\end{rem}

\subsection{Explicit Formulas}
The various group operations can also be expressed in explicit formulas, which will be explained in this section.
\begin{thm}
	Let $E$ be an elliptic curve given by the Weierstrass equation
	\begin{equation*}
	f(x,y)=y^2+a_1xy+a_3y-x^3-a_2x^2-a_4x-a_6=0,
	\end{equation*}
	\text\linebreak
	\begin{enumerate}
		\item Let $P_0=(x_0,y_0)\in E$. Then
		\begin{equation*}
		-P_0=(x_0,-y_0-a_1x_0-a_3).
		\end{equation*}
		Now let $P_1+P_2=P_3$ with $P_i=(x_i,y_i)\in E$.
		\item If $x_1=x_2$ and $y_1+y_2+a_1x_2+a_3=0$ then
		\begin{equation*}
		P_1+P_2=0.
		\end{equation*}
		If this is not the case, define for $x_1\neq x_2$
		\begin{equation*}
		\lambda=\frac{y_2-y_1}{x_2-x_1}
		\end{equation*}
		\begin{equation*}
		\nu=\frac{y_1x_2-y_2x_1}{x_2-x_1}
		\end{equation*}
		and for $x_1=x_2$
		\begin{equation*}
		\lambda=\frac{3x_1^2+2a_2x_1+a_4-a_1y_1}{2y_1+a_1x_1+a_3}
		\end{equation*}
		\begin{equation*}
		\nu=\frac{-x_1^3+a_4x_1+2a_6-a_3y_1}{2y_1+a_1x_1+a_3}
		\end{equation*}
		\\
		Then $y=\lambda x+\nu$ is the line through $P_1$ and $P_2$.
		\item Using the same notation as $2.$, the coordinates for $P_3=P_1+P_2$ are $(x_3,y_3)$ with
		\begin{equation*}
		x_3=\lambda^2+a_1\lambda-a_2-x_1-x_2
		\end{equation*}
		\begin{equation*}
		y_3=-(\lambda+a_1)x_3-\nu-a_3
		\end{equation*}
	\end{enumerate}
\end{thm}
\begin{prf}
	This theorem is verified by using basic algebra on the equations for the various lines, and is found in \cite[pages 52-54]{EllipticCurvesBook}. Note the following:
	\begin{enumerate}
		\item A line $L$ is drawn through $O$ and $P$. The line is given by $L:x-x_0=0$, which with substitution in $E$ results in $-P_0$.
		\item If $P_2=-P_1$, then the first item is the proof. Otherwise the line $L$ through $P_1$ and $P_2$ has an equation of the form $L:y=\lambda x+\nu$, with $\lambda$ the slope and $\nu$ the value of the line at $x=0$. The first values for $\lambda$ and $\nu$ are for distinct points, in the second case the line is the tangent at $P_1=P_2$. In all cases the line intersects $E$ at $O$.
		\item Here the same line $L$ can be used as in the previous point, but the third point of intersection is now not $O$, but another point $P_3$ on $E$.
	\end{enumerate}
\end{prf}

\subsection{Elliptic Curves}
This part will focus on linking Weierstrass equations to generic elliptic curves, to show that the results achieved for Weierstrass equations apply generally.
\begin{defn}
	An elliptic curve is a pair $(E,O)$, with $E$ a non-singular curve of genus one, and $O\in E$. The elliptic curve $E$ is defined over $K$, written as $E/K$, if $E$ is defined over $K$ and $O\in E(K)$.
\end{defn}

To show that elliptic curves and Weierstrass equations are linked the Riemann-Roch theorem is used (see \ref{RiemannRoch}).


\begin{thm}
	Let $E$ be an elliptic curve defined over $K$, then the following holds:
	\begin{enumerate}
		\item There exist functions $x,y\in K(E)$ such that the map
		\begin{equation*}
		\phi:e\to\mathbb{P}^2\text{ , }\phi=[x,y,1]
		\end{equation*}
		gives an isomorphism of $E/K$ onto a curve given by the Weierstrass equation
		\begin{equation*}
		C:Y^2+a_1XY+a_3Y=X^3+a_2X^2+a_4X+a_6
		\end{equation*}
		satisfying $\phi(O)=[0,1,0]$. The functions $x$ and $y$ are called the Weierstrass coordinates for the elliptic curve $E$
		\item Any two Weierstrass equations for $E$ as in $1.$ are related by a linear change of variables of the form
		\begin{equation*}
		X=u^2X'+r
		\end{equation*}
		\begin{equation*}
		Y=u^3Y'+su^2X'+t
		\end{equation*}
		where $u\in K^*$ and $r,s,t\in K$.
		\item Conversely, every smooth cubic curve $C$ given by a Weierstrass equation as in $1.$ is an elliptic curve defined over $K$ with base point $O=[0,1,0]$.
	\end{enumerate}
\end{thm}

It is easy to see that points $1.$ and $2.$ combined make it so that the functions map uniquely to Weierstrass equations, and combining $1.$ and $3.$ shows that, indeed, every Weierstrass equation is linked to an elliptic curve, and that the inverse is also true.

An overview of the structure of the proof for point $1.$ of theorem (2.3.4.2) will be given below, the complete proof is found in \cite[pages 59-60]{EllipticCurvesBook}.
\begin{prf}
	By looking at vector spaces $\mathcal{L}(n(O))$ and using the Riemann-Roch theorem for genus $1$, it is easy to see that $\ell(n(O))=dim(\mathcal{L}(n(O)))=n$ for $n\in\mathbb{N}$. Therefore functions $x,y\in K(E)$ can be chosen such that $\{1,x\}$ is a basis for $\mathcal{L}(2(O))$ and $\{1,x,y\}$ is a basis for $\mathcal{L}(3(O))$. For this, $x$ must have a pole of exact order $2$ at $O$, and $y$ must have a pole of exact order $3$ at $O$.
	
	Observe that $Dim(\mathcal{L}(6(O)))=6$, but that it contains seven functions in total: $1,x,y,x^2,xy,y^2,x^3$. As such there is a linear relation
	\begin{equation}
	A_1+A_2x+A_3y+A_4x^2+A_5xy+A_6y^2+A_7x^3=0
	\end{equation}
	Here $A_6,A_7\neq0$, for otherwise every term would have a pole at $O$ of a different order, which means that all $A_i$ would disappear.
	
	Substituting $(x,y)\to(-A_6A_7x,A_6A_7y)$ in (2.15) leads to the following equation in Weierstrass form
	\begin{equation}
	A_1-A_2A_6A_7x+A_3A_6A_7^2y+A_4A_6^2A_7^2x^2-A_5A_6^2A_7^3xy+A_6^{3}A_7^{4}y^2-A_2A_6^{3}A_7^{4}x^3=0
	\end{equation}
	
	Dividing (2.16) by $A_6^3A_7^4$ leads to an equation in Weierstrass form
	\begin{equation}
	y^2+A_5A_6^{-1}A_7^{-1}xy+A_3A_6^{-2}A_7^{-2}y=x^3-A_4A_6^{-1}A_7^{-2}x^2+A_2A_6^{-2}A_7^{-3}x-A_1
	\end{equation}
	Equation (2.17) leads to the map
	\begin{equation*}
	\phi:E->\mathbb{P}^2\text{ ,	}\phi=[x,y,1]
	\end{equation*}
	of which the image lies on $C$ as described by the Weierstrass equation.
	
	The proof goes on to show that this map is an isomorphism, by noting that $\phi(O)=[0,1,0]$ and that the map is of degree one. Showing that $C$ is smooth allows map composition with a map $\psi:C\to\mathbb{P}^1$ of degree one, which allows to show that $\phi$ is an isomorphism.
\end{prf}
In short, it can be proven that a map can be constructed between Weierstrass equations and elliptic curves in such a manner that the map is an isomorphism. This, in turn, means that there is an elliptic curve for every Weierstrass equation and conversely, that there is a Weierstrass equation for each elliptic curve.

The result of this is that the group law, previously proven for elliptic curves in the form of the Weierstrass equation, applies to all elliptic curves, and that any elliptic curve therefore forms an abelian group.

\subsection{Weil Pairing}
This pairing uses $E_m$, the group of $m$-torsion points and $\mu_m$, the group of $m$-th roots of unity.
\begin{defn}[$m$-torsion points]
	$m$-torsion points are points $P$ on an elliptic curve $E$ such that $mP=O$
	\begin{equation*}
	E[m]=\{P:mP=O\}
	\end{equation*}
\end{defn}
\begin{defn}[$m$-th roots of unity]
	$m$-th roots of unity are complex numbers $c$ such that $c^m=1$
	\begin{equation*}
	\mu_m=\{c\in\mathbb{C}:c^m=1\}
	\end{equation*}
\end{defn}
The Weil pairing also uses functions $f\in\bar{K}(E)$ and $g\in\bar{K}(E)$ which satisfy the following.
\begin{defn}
	Given $T\in E[m]$ there is a function $f\in\bar{K}(E)$ such that
	\begin{equation}
	div(f)=m(T)-m(O)
	\end{equation}
	Given $T'\in E$ to be a point such that $[m]T'=T\in E[m]$ there is a function $g\in\bar{K}(E)$ such that
	\begin{equation}
	div(g)=\sum_{R\in E[m]}((T'+R)-(R))
	\end{equation}
	Then it follows that given $S\in E[m]$, for any point $X\in E$ the following holds:
	\begin{equation}
	g(X+S)^m=f([m]X+[m]S)=f([m]X)=g(X)^m
	\end{equation}
	Here $f([m]X+[m]S)=f([m]X)$ for $S$ is a $m$-torsion point, so $[m]S=O$.
\end{defn}
\text\linebreak
\\
Using function $g$ such that is satisfies (2.19) allows to construct the Weil pairing:
\begin{defn}
	Setting, for $X\in E$ and $g\in\bar{K}(E)$
	\begin{equation} \label{wp}
	E_m(S,T)=\frac{g(X+S)}{g(X)}
	\end{equation}
	the Weil pairing is a non-degenerate alternating bilinear form
	\begin{equation*}
	e_m=E_m\times E_m\to\mu_m
	\end{equation*}
\end{defn}
This pairing has the following properties
\begin{thm}
	The Weil $e_m$ pairing has these properties
	\begin{enumerate}
		\item The pairing is bilinear
		\begin{equation*}
		e_m(S_1+S_2,T)=e_m(S_1,T)e_m(S_2,T),
		\end{equation*}
		\begin{equation*}
		e_m(S,T_1+T_2)=e_m(S,T_1)e_m(S,T_2),
		\end{equation*}
		\item The pairing is alternating
		\begin{equation*}
		e_m(T,T)=1
		\end{equation*}
		which means in particular that
		\begin{equation*}
		e_m(S,T)=e_m(T,S)^{-1}
		\end{equation*}
		\item The pairing is non-degenerate
		\begin{equation*}
		\text{If } e_m(S,T)=1\text{ for all } S\in E_m\text{, then } T=O
		\end{equation*}
		\item The pairing is Galois invariant
		\begin{equation*}
		e_m(S,T)^\sigma=e_m(S^\sigma,T^\sigma)\text{ for all }\sigma\in G_{\bar{K}/K}
		\end{equation*}
		\item The pairing is compatible
		\begin{equation*}
		e_{mm'}(S,T)=e_m(m'S,T)\text{ for all }S\in E_{mm'}\text{ and }T\in E_m
		\end{equation*}
	\end{enumerate}
\end{thm}

\begin{prf}
	The complete proof is found in \cite[page 96]{EllipticCurvesBook}, but note that 2.5.5.5.1 is easily verified with (\ref{wp}) by letting $f_1,f_2,f_3,g_1,g_2,g_3$ be the appropriate functions for $T_1+T_2=T_3$ and choosing a function $h$ such that
	\begin{equation*}
	div(h)=(T_1+T_2)-(T_1)-(T_2)+(O)
	\end{equation*}
\end{prf}

\section{Discrete Logarithm Problem}
On the set of real numbers ($\mathbb{R}$) the $\log_b$ function has been defined as the solution to the following problem.
\begin{defn}
	Given $a,b,n\in\mathbb{R}$, base $b$ and power $a$ of $b$, what is $n$ such that $a=b^n$?
\end{defn}
This same problem can also be defined over modulo $p$, which is known as the discrete logarithm problem over $\mathbb{Z}_p$.
\begin{defn}[Discrete Logarithm Problem over $\mathbb{Z}_p$]
	Given $a,b\in\mathbb{Z}_p$, base $b$ and power $a$ of $b$, what is $n$ such that $a=b^n\mod{p}$?
\end{defn}
In a more generic sense, this problem can be defined over a group $G$.
\begin{defn}[Discrete Logarithm Problem over group $G$]
	Given $a,b\in G$, base $b$ and power $a$ of $b$, what is $n$ such that $a=b^n$?
\end{defn}
In particular, since an elliptic curve $E$ is a group, the problem holds over elliptic curves. Taking elliptic curve $E$ as group $G$, it follows that.
\begin{defn}[Elliptic Curve Discrete Logarithm Problem]
	Given elliptic curve $E$ and points $P,Q\in E$, what is $n$ such that $nP=Q$?
\end{defn}
Although no actual proof exists, the assumption is that the discrete logarithm problem in a well chosen group $G$ is a hard problem. The group must be well chosen, for there are groups that have a structure that allows for an algorithm to solve the problem, but there is a sufficiently large group of groups left for which no such algorithm exists.

\subsection{Applications in Cryptography}
Because there is no efficient algorithm to solve the problem, anyone trying to find a solution to the problem, must try various solutions. This means that the runtime of an algorithm for finding a solution grows linearly to the group size, which is equivalent to saying that it grows exponentially in the amount of digits of the group size.

The hardness of the problem leads to applications in cryptography. By using a group that is large enough that trying all solutions becomes infeasible, data can be secured with the help of cryptographic schemes that are based on the (elliptic curve) discrete logarithm problem.

\section{Schnorr Group} \label{SchnorrGroup}
A Schnorr group \cite{Schnorr}, proposed by Claus P. Schnorr, inventor of the Schorr Signature Scheme, is another group that find use in cryptographic schemes. Specifically in Schnorr-based schemes, as discussed in (3.6.3). The definition for the Schnorr group is as follows.
\begin{defn}[Schnorr Group]
	Generate $p,q,r$ such that $p=qr+1$ with $p,q$ prime. Then pick any $1<h<p$ such that $h^r\not\equiv1\mod{p}$. Then $g=h^r\mod{p}$ is the generator of the Schnorr group, which is a subgroup of $\mathbb{Z}_{p}^*$ of order $q$
\end{defn}

\begin{prf}[Schnorr Group]
	It is trivial to see that the Schnorr group is indeed a group. Note that the order of $\mathbb{Z}_{p}^*$ is $p-1=qr$. Because $\mathbb{Z}_{p}^*$ is cyclic, for each divisor $d$ of $qr$ there is one subgroup of order $d$, generated by $a^{n/d}$, with $a\in\mathbb{Z}_{p}^*$. As such there is a subgroup of order $q$ generated by $a^{qr/q}$, which is precisely the $h$ picked. As such the order of the Schnorr group is indeed $q$.
\end{prf}