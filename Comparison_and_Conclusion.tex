\chapter{Comparison and Conclusion}
This chapter will compare the BLS (see \ref{BLSMulti}) and MuSig (see \ref{SchnorrMulti}) multi-signature algorithms and propose the best solution to improve OmniLedger. The performance will briefly touch upon the security requirements of both schemes, but focuses mostly on the performance of the schemes. The focus of this thesis is, in the end, on improving the performance of OmniLedger to ensure even better scalability.

\section{Comparison}
Both of the algorithms use the discrete logarithm problem, or a related problem, to prove the security of the scheme. Both proofs of security show that, in order for the attacker to successfully attack the system, he must have found an algorithm that can be used to solve the discrete logarithm problem in the case of MuSig, or one that can be used to solve the computational Diffie-Hellman problem in the case of the BLS multi-signature scheme. Because both schemes use, at the basis, the same proof of security, the differences in security are marginal at best.

Considering the computational requirements of both schemes, it is easy to see that they operate in a similar manner. Key generation, signing and verification require basic computational actions that require very little in terms of computational power and the computational requirements do therefore not significantly impact the performance of either scheme. It must be noted that the BLS multi-signature algorithm allows for aggregation of the public key even before the message that must be signed is known, which allows for some performance optimisation. Furthermore, the BLS multi-signature scheme also allows for batches of multi-signature algorithms to be checked at the same time, allowing for even more optimisation.

However, in the part of the algorithm where a message is signed, a rather significant performance discrepancy is found. MuSig requires three round-trips of communication during this part of the scheme, whereas BLS only uses one. Moreover, MuSig requires that all participants are on-line when creating the multi-signature, since the three round-trips each require in- and output from all participants before the algorithm can continue. This means that each round of communication not only impacts the performance by the way of network latency, but can also be the cause for critical failure if the network for one or more participants fails at any point during the message signing. Contrary, BLS can operate even under unstable networks, since it allows all steps barring the aggregation of the final signature to be undertaken without communication. This means that all participants in the BLS multi-signature scheme can produce their part of the signature when they are able to do so, and that the aggregated signature can be computed as soon as every participant has published his part. This means that the network latency only impacts the algorithm only once, and that network blackout does not cause critical failure.

\section{Conclusion}
Because both schemes use a very similar proof of security and remain similar in terms of computational requirements, it suffices to compare the performance directly to pick the most suitable multi-signature scheme.

The BLS multi-signature scheme requires only one round-trip of communication, minimising the performance loss because of network latency, and does not suffer critical failure upon network blackout during the signing process. Since the scheme also offers performance optimisation, it is the superior choice for a multi-signature algorithm to improve the performance of RandHound and CoSi, thereby improving the performance of OmniLedger.