\chapter{Related Work}
In this chapter various related works regarding multi-signature algorithms and their contributions are discussed.

The work on OmniLedger \cite{OmniLedger} by Eleftherios Kokoris-Kogias, Philipp Jovanovic, Linus Gasser, Nicolas Gailly, Ewa Syta and Bryan Ford uses CoSi \cite{CoSi} in the consensus protocol. CoSi uses a Schnorr-based multi-signature scheme, and OmniLedger has used CoSi as such, not looking for improvement on the algorithm.

Ewa Syta, Philipp Jovanovic, Eleftherios Kokoris Kogias, Nicolas Gailly, Linus Gasser, Ismail Khoffi, Michael J. Fischer and Bryan Ford in \cite{RandHound} state that they rely on Schnorr-based multi-signature schemes, but do not give options or arguments for other multi-signature schemes. As such their work simply shows that Schnorr-based multi-signature schemes are used in other applications, but it gives no indication towards the performance in comparison to other multi-signature schemes.

Ewa Syta, Iulia Tamas, Dylan Visher, David Isaac Wolinsky, Philipp Jovanovic, Linus Gasser, Nicolas Gailly, Ismail Khoffi and Bryan Ford \cite{CoSi} in their work on CoSi have stated that, although they chose for a Schnorr-based multi-signature scheme, any multi-signature scheme with efficient public key and signature aggregation could be used. The choice for a Schnorr-based scheme was made since such a scheme is simple and well-understood. They did note that a BLS \cite{BLS} based scheme might be more desirable in a unstable or asynchronous situations, but they did not focus on the performance of either scheme.

In their work on the BLS multi-signature scheme \cite{BLSMulti}, Dan Boneh, Manu Drijvers and Gregory Neven briefly compare their scheme to Schnorr-based multi-signature schemes, but their comparison does not focus on performance. It is mentioned that their scheme, as opposed to Schnorr-based schemes, does not require multiple rounds of communication, but that is the only comparison regarding performance. They mention that their scheme allows for public aggregation via simple multiplication long after signatures have been generated,  as opposed to the Schnorr-based multi-signatures that can only be aggregated at the time of signing. Furthermore they give applications of their scheme in crypto currencies, but with a focus on shrinking the transaction size by limiting the size of the signature, not regarding performance.

Gregory Maxwell, Andrew Poelstra, Yannick Seurin and Pieter Wuille in \cite{SchnorrMulti} describe a Schnorr-based multi-signature scheme that is both simple and efficient, and allows for key aggregation to create a joint signature. They compare their work to existing Schnorr-based multi-signature algorithms to show that their algorithm is an improvement in regards to performance and simplicity.

The contribution of this thesis, is that it compares two existing multi-signature algorithms, BLS multi-signature \cite{BLSMulti} and MuSig \cite{SchnorrMulti} with a focus on performance. This contributes to any future works where the performance of a multi-signature scheme is an important aspect.